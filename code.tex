% Options for packages loaded elsewhere
\PassOptionsToPackage{unicode}{hyperref}
\PassOptionsToPackage{hyphens}{url}
%
\documentclass[
]{article}
\usepackage{amsmath,amssymb}
\usepackage{lmodern}
\usepackage{iftex}
\ifPDFTeX
  \usepackage[T1]{fontenc}
  \usepackage[utf8]{inputenc}
  \usepackage{textcomp} % provide euro and other symbols
\else % if luatex or xetex
  \usepackage{unicode-math}
  \defaultfontfeatures{Scale=MatchLowercase}
  \defaultfontfeatures[\rmfamily]{Ligatures=TeX,Scale=1}
\fi
% Use upquote if available, for straight quotes in verbatim environments
\IfFileExists{upquote.sty}{\usepackage{upquote}}{}
\IfFileExists{microtype.sty}{% use microtype if available
  \usepackage[]{microtype}
  \UseMicrotypeSet[protrusion]{basicmath} % disable protrusion for tt fonts
}{}
\makeatletter
\@ifundefined{KOMAClassName}{% if non-KOMA class
  \IfFileExists{parskip.sty}{%
    \usepackage{parskip}
  }{% else
    \setlength{\parindent}{0pt}
    \setlength{\parskip}{6pt plus 2pt minus 1pt}}
}{% if KOMA class
  \KOMAoptions{parskip=half}}
\makeatother
\usepackage{xcolor}
\usepackage[margin=1in]{geometry}
\usepackage{graphicx}
\makeatletter
\def\maxwidth{\ifdim\Gin@nat@width>\linewidth\linewidth\else\Gin@nat@width\fi}
\def\maxheight{\ifdim\Gin@nat@height>\textheight\textheight\else\Gin@nat@height\fi}
\makeatother
% Scale images if necessary, so that they will not overflow the page
% margins by default, and it is still possible to overwrite the defaults
% using explicit options in \includegraphics[width, height, ...]{}
\setkeys{Gin}{width=\maxwidth,height=\maxheight,keepaspectratio}
% Set default figure placement to htbp
\makeatletter
\def\fps@figure{htbp}
\makeatother
\setlength{\emergencystretch}{3em} % prevent overfull lines
\providecommand{\tightlist}{%
  \setlength{\itemsep}{0pt}\setlength{\parskip}{0pt}}
\setcounter{secnumdepth}{-\maxdimen} % remove section numbering
\usepackage{booktabs}
\usepackage{longtable}
\usepackage{array}
\usepackage{multirow}
\usepackage{wrapfig}
\usepackage{float}
\usepackage{colortbl}
\usepackage{pdflscape}
\usepackage{tabu}
\usepackage{threeparttable}
\usepackage{threeparttablex}
\usepackage[normalem]{ulem}
\usepackage{makecell}
\usepackage{xcolor}
\ifLuaTeX
  \usepackage{selnolig}  % disable illegal ligatures
\fi
\IfFileExists{bookmark.sty}{\usepackage{bookmark}}{\usepackage{hyperref}}
\IfFileExists{xurl.sty}{\usepackage{xurl}}{} % add URL line breaks if available
\urlstyle{same} % disable monospaced font for URLs
\hypersetup{
  pdftitle={Etude des violences sexuellles par département en France},
  pdfauthor={Bill Yehouenou, Jeanne Ropert},
  hidelinks,
  pdfcreator={LaTeX via pandoc}}

\title{Etude des violences sexuellles par département en France}
\author{Bill Yehouenou, Jeanne Ropert}
\date{2023-01-19}

\begin{document}
\maketitle

{
\setcounter{tocdepth}{2}
\tableofcontents
}
\newpage

\hypertarget{probluxe9matique}{%
\section{Problématique}\label{probluxe9matique}}

Les violences sexuelles se définissent comme « \emph{toute atteinte
sexuelle commise sans le consentement d'une personne ou tout agissement
discriminatoire fondé sur le sexe} ». L'évolution des mentalités et de
la loi a permis ces dernières années, une libération de la parole. En
effet, le nombre de violences sexuelles recensées est en hausse de 33\%
en 2021. De plus, selon une enquête publiée en 2021 par l'INSERM, «
\textbf{14,5 \% des femmes et 6,4 \% des hommes en France, soit environ
5,5 millions de personnes, auraient été confrontés avant l'âge de 18 ans
à des violences sexuelles} ». Cette dernière nous apprend également que
ces violences apparaissent dans la majeure partie des cas dans un cadre
familial. Nous nous sommes questionnées sur les facteurs qui pourraient
influencer le nombre de violences sexuelles, en analysant les violences
sexuelles physiques par département de France métropolitaine en 2018. En
effet, \textbf{comment expliquer les différences du nombre de faits de
violences sexuelles entre les départements ?}. Ainsi, l'équation de
notre modèle devrait se présenter comme suit :

\[ Faits=b_0+b_1*Mediane.revenu+b_2*Pop+b_3*Homme.sans.diplome+b_4*Femme.sans.diplome+\]
\[ b_5*Taux.chomage+b_6*Taux.pauvrete+b_7*Taux.logements.sociaux+b_8*Sexe.politique+b_9*Geographie+e \]

\hypertarget{collecte-des-donnuxe9es}{%
\section{Collecte des données}\label{collecte-des-donnuxe9es}}

Pour commencer notre étude, nous avons constitué notre base de données,
pour cela nous avons sélectionné neuf variables explicatives et une
variable à expliquer. Notre variable endogène est donc le \emph{nombre
de faits de violences sexuelles physiques par département}.

Concernant, nos variables exogènes, nous nous sommes d'abord intéressées
au niveau de vie des départements. Pour cela, nous avons réuni des
données concernant \emph{le revenu médian disponible en euros}, \emph{le
taux de pauvrete}, \emph{le taux de chômage} et \emph{le taux de
logements sociaux}. Également, nous avons voulu faire intervenir
quelques caractéristiques sur la population, comme son nombre ou encore
les effectifs des hommes et femmes de plus de 25 ans sans diplômes. Pour
finir, nous nous sommes penchées sur deux caractéristiques des
départements, notamment \emph{la situation géographique}, codée 0 pour
le nord et 1 pour le sud, et \emph{le sexe du président du conseil
départemental}, codé 0 pour les hommes et 1 pour les femmes. Nous avons
recueilli ces données concernant 96 départements sur le site du
gouvernement, de l'Insee et de l'observatoire des territoires.

\begin{verbatim}
# A tibble: 10 x 4
   Code                                   Définition      Nature `Signe attendu`
   <chr>                                  <chr>           <chr>  <chr>          
 1 Faits                                  Mediane du rev~ "Quan~ +              
 2 Mediane_revenu_dispo                   Population tot~ "Quan~ +              
 3 Homme_sans_diplome                     Effectif des h~ "Quan~ +              
 4 Femme_sans_diplome                     Effectif des f~ "Quan~ +              
 5 Taux_chomage                           Taux de chômag~ "Quan~ +              
 6 Taux_pauvrete                          Taux de pauvre~ "Quan~ +              
 7 Taux_logements_sociaux                 Taux de logeme~ "Quan~ +              
 8 Sexe_politique                         Sexe personali~ "Indi~ +/- selon le c~
 9 Geographie                             Situation géog~ "Indi~ +/- selon le c~
10 Nombre de faits de violences pour 1000 Quantitative    ""     Faits          
\end{verbatim}

\hypertarget{partie-1-etude-de-lhuxe9tuxe9roscuxe9dasticituxe9}{%
\section{Partie 1 : Etude de
l'hétéroscédasticité}\label{partie-1-etude-de-lhuxe9tuxe9roscuxe9dasticituxe9}}

\hypertarget{statistiques-descriptives-univariuxe9es}{%
\subsection{Statistiques descriptives
univariées}\label{statistiques-descriptives-univariuxe9es}}

Pour en apprendre davantage sur notre variable d'intérêt, nous allons
procéder à une analyse statistique desctriptive. La boîte à moustache de
notre variable d'intérêt représente une \textbf{variabilité assez
faible} mais avec des valeurs atypiques notables.

\includegraphics{code_files/figure-latex/variable interet-1.pdf}

L'analyse de l'histogramme du nombre de faits montre que les
observations sont assez concentrées entre 0 et 1 fait pour 1000
habitant. Toutefois, elle possède quelques outliers qu'il faudra
surveiller. Nous avons décidé de nous intéresser à la variable
\emph{taux de logements sociaux} qui, d'àprès notre revue de
littérature, est une variable assez importante.

\includegraphics{code_files/figure-latex/histogramme logement-1.pdf}

L'histogramme du \emph{taux de logements sociaux} révèle des valeurs
atypiques. De plus, on constate que plusieurs départements sont en
dessous du quota de logements sociaux prévus en France, soit 10\%. En
termes de distribution, la variable du \emph{taux de logement sociaux},
sa distribution semble asymétrique ce qui peut présenter un problème
pour notre modèle.

\hypertarget{statistiques-descriptives-bivariuxe9es}{%
\subsection{Statistiques descriptives
bivariées}\label{statistiques-descriptives-bivariuxe9es}}

Pour en apprendre davantage sur le type de lien qui existe entre notre
variable endogène et nos variables exogènes, nous utilisons la
représentation graphique en nuage de points. Tout d'abord, comme nous le
montre les graphiques ci-dessous, la médiane du revenu disponible, le
taux de pauvrete et le nombre d'hommes et femmes sans diplômes sont
négativement et linéairement corrélés à notre variable endogène,
c'est-à-dire le nombre de violences sexuelles.

\includegraphics{code_files/figure-latex/analyse bivariee-1.pdf}

Ainsi, cette première partie nous a permis de visualiser l'ensemble de
notre base de données, de constater les éventuels problèmes, les
corrélations et le type de lien entre les variables. Cette étape est
nécessaire pour poursuivre au mieux la réalisation du modèle
économétrique.

\hypertarget{matrice-de-corruxe9lation}{%
\subsection{Matrice de corrélation}\label{matrice-de-corruxe9lation}}

Désormais, nous allons procéder à une analyse statistique descriptive
bivariée afin d'étudier les relations entre le nombre de violences
sexuelles recensées par département et nos variables explicatives, mais
également les possibles liens entre nos variables explicatives
elles-mêmes.

\includegraphics{code_files/figure-latex/matrice de correlation-1.pdf}

De notre côté, les faits de violences sexuelles sont fortement corrélés
au nombre de femmes et d'hommes de plus de 25 ans sans diplômes, et plus
légèrement corrélés à la médiane du revenu disponible et au taux de
logements sociaux. Par ailleurs, cette matrice met en évidence des
corrélations entre des variables explicatives. En effet, les variables
Homme et Femme sans diplômes ou encore Revenu disponible et Taux de
pauvreté sont corrélées entre elles. Une présence de colinéarité entre
des variables explicatives peut altérer notre modèle, un choix devra
donc être fait au moment de la régression.

\hypertarget{moduxe8le}{%
\subsection{Modèle}\label{moduxe8le}}

\hypertarget{spuxe9cification-du-moduxe8le}{%
\subsubsection{Spécification du
modèle}\label{spuxe9cification-du-moduxe8le}}

Afin d'estimer le nombre de faits de violences sexuelles par
département, nous devons mettre en place un modèle économétrique
constitué des variables qui pourraient expliquer ces violences. Pour
choisir le modèle qui estimera le mieux les violences sexuelles, nous
commençons par effectuer une régression linéaire du nombre de faits par
département, en fonction des neuf autres variables

\begin{verbatim}

==========================================================================================
                                               Dependent variable:                        
                       -------------------------------------------------------------------
                                                      Faits                               
                                (1)                    (2)                    (3)         
------------------------------------------------------------------------------------------
Mediane_revenu_dispo          -0.00000             -0.00000***                            
                             (0.00000)              (0.00000)                             
                                                                                          
Homme_sans_diplome           -0.00000*                                                    
                             (0.00000)                                                    
                                                                                          
Femme_sans_diplome             0.000                                                      
                             (0.00000)                                                    
                                                                                          
Taux_chomage                 -0.0004***             -0.001***             -0.0005***      
                              (0.0001)               (0.0001)              (0.0001)       
                                                                                          
Taux_pauvrete                0.0003***              0.0002***              0.0003***      
                              (0.0001)               (0.0001)              (0.0001)       
                                                                                          
Taux_logements_sociaux        0.00004               -0.0001**             -0.0001***      
                             (0.00003)              (0.00002)              (0.00003)      
                                                                                          
Sexe_politique1                0.0005                                                     
                              (0.0003)                                                    
                                                                                          
Geographie1                    0.0003                                       -0.0001       
                              (0.0003)                                     (0.0003)       
                                                                                          
Constant                      0.006**                0.012***              0.003***       
                              (0.003)                (0.003)                (0.001)       
                                                                                          
------------------------------------------------------------------------------------------
Observations                     96                     96                    96          
R2                             0.576                  0.349                  0.271        
Adjusted R2                    0.537                  0.321                  0.239        
Residual Std. Error       0.001 (df = 87)        0.001 (df = 91)        0.001 (df = 91)   
F Statistic            14.766*** (df = 8; 87) 12.206*** (df = 4; 91) 8.460*** (df = 4; 91)
==========================================================================================
Note:                                                          *p<0.1; **p<0.05; ***p<0.01
\end{verbatim}

Après avoir mis en route et comparé différents modèles, nous avons
retenu un modèle (1) avec toutes les variables. En effet, il s'agit du
modèle avec le meilleur R2 ajusté. Il faudra faire un test de Ramsey
pour vérifier la spécification du modèle.

\hypertarget{test-de-ramsey}{%
\subsubsection{Test de Ramsey}\label{test-de-ramsey}}

\begin{verbatim}

    RESET test

data:  model1
RESET = 25.57, df1 = 2, df2 = 85, p-value = 2.023e-09
\end{verbatim}

La p-value du test de Ramsey est inférieure au seuil de 5\% alors, le
modèle est bien spécifié.

\hypertarget{analyse-des-ruxe9sidus-du-moduxe8le}{%
\subsection{Analyse des résidus du
modèle}\label{analyse-des-ruxe9sidus-du-moduxe8le}}

\hypertarget{ruxe9sidus-studentisuxe9s}{%
\subsubsection{Résidus studentisés}\label{ruxe9sidus-studentisuxe9s}}

Les résidus studentisés obtenus par validation croisée nous montrent que
les départements cités ci-dessous sont des départements atypiques dans
notre base de données.

\begin{verbatim}
[1] "Alpes-de-Haute-Provence" "Hautes-Alpes"           
[3] "Cantal"                  "Lozère"                 
[5] "Nord"                    "Territoire de Belfort"  
\end{verbatim}

On a procédé aussi à un lissage de nos résidus et ce lissage est assez
proche de 0. Donc, on soupçonne que nos résidus ont une espérance nulle
et de variance constante.

\includegraphics{code_files/figure-latex/residus studentisees-1.pdf}

\hypertarget{test-de-normalituxe9-des-ruxe9sidus-de-shapiro-wilk}{%
\subsubsection{Test de normalité des résidus de
Shapiro-Wilk}\label{test-de-normalituxe9-des-ruxe9sidus-de-shapiro-wilk}}

\begin{verbatim}

    Shapiro-Wilk normality test

data:  model1$residuals
W = 0.91549, p-value = 1.197e-05
\end{verbatim}

La p-value du test de normalité des résidus est inférieure au seuil de
5\% alors on rejette l'hypothèse de normalité des résidus. Les résidus
ne suivent pas une loi normale. Cela est dû à l'existence de valeurs
atypiques qui étendent la distribution.

\includegraphics{code_files/figure-latex/graphique de normalite-1.pdf}

\hypertarget{test-dhuxe9tuxe9roscuxe9dasticituxe9}{%
\subsubsection{Test
d'hétéroscédasticité}\label{test-dhuxe9tuxe9roscuxe9dasticituxe9}}

\begin{verbatim}

    studentized Breusch-Pagan test

data:  model1
BP = 20.766, df = 8, p-value = 0.007795
\end{verbatim}

La p-value est inférieure au seuil de 5\% alors on rejette l'hypothèse
nulle d'homoscédasticité. Donc, on est en présence d'un problème
d'hétéroscédasticité des résidus. Il faut donc appliquer la correction
de White.

\hypertarget{correction-de-white}{%
\subsection{Correction de White}\label{correction-de-white}}

\begin{verbatim}

t test of coefficients:

                          Estimate  Std. Error t value  Pr(>|t|)    
(Intercept)             5.5206e-03  2.9537e-03  1.8691 0.0649777 .  
Mediane_revenu_dispo   -1.5250e-07  1.0004e-07 -1.5244 0.1310443    
Homme_sans_diplome     -5.2270e-08  4.4386e-08 -1.1776 0.2421598    
Femme_sans_diplome      9.7135e-09  4.3406e-08  0.2238 0.8234497    
Taux_chomage           -4.3372e-04  1.6051e-04 -2.7022 0.0082813 ** 
Taux_pauvrete           2.6370e-04  7.4535e-05  3.5379 0.0006497 ***
Taux_logements_sociaux  3.5063e-05  2.5560e-05  1.3718 0.1736575    
Sexe_politique1         4.9341e-04  4.8360e-04  1.0203 0.3104212    
Geographie1             3.0822e-04  2.9211e-04  1.0551 0.2942932    
---
Signif. codes:  0 '***' 0.001 '**' 0.01 '*' 0.05 '.' 0.1 ' ' 1
\end{verbatim}

Après la mise en oeuvre de la méthode, on remarque que le niveau de
significativité des coefficients a diminué car le modèle a été corrigée
de l'hétéroscédasticité. Toutefois, les variables significatives
demeurent toujours significatives au seuil de 5\%.

\hypertarget{partie-2-analyse-de-la-multicolinuxe9arituxe9}{%
\section{Partie 2 : Analyse de la
multicolinéarité}\label{partie-2-analyse-de-la-multicolinuxe9arituxe9}}

\hypertarget{duxe9tection-de-la-multicolinuxe9arituxe9}{%
\subsection{Détection de la
multicolinéarité}\label{duxe9tection-de-la-multicolinuxe9arituxe9}}

Afin de confimer nos soupçons concernant l'existence de multicolinéarité
au sein de notre modèle final, nous regardons les facteurs d'inflation
de la variance (VIF).

\begin{verbatim}
  Mediane_revenu_dispo     Homme_sans_diplome     Femme_sans_diplome 
              2.516668              50.418393              47.280059 
          Taux_chomage          Taux_pauvrete Taux_logements_sociaux 
              2.759233               4.116913               2.568499 
        Sexe_politique             Geographie 
              1.124385               1.782771 
\end{verbatim}

Le \textbf{VIF} calcule la colinéarité d'une variable en fonction des
autres régresseurs. La valeur au-dessus duquel nous considérons qu'il y
a de la multicolinéarité n'est pas fixe, nous prendrons donc \textbf{5}
comme valeur de reférence. On est donc en présence de multicolinéarité
pour les variables \texttt{Homme} et \texttt{Femme} sans diplôme.

\hypertarget{muxe9thodes-de-ruxe9duction-de-dimension}{%
\subsection{Méthodes de réduction de
dimension}\label{muxe9thodes-de-ruxe9duction-de-dimension}}

\hypertarget{partitionnement-des-donnuxe9es}{%
\subsection{Partitionnement des
données}\label{partitionnement-des-donnuxe9es}}

Nous construirons notre modèle sur les données d'entraînement et
évaluerons ses performances sur les données de test. Il s'agit d'une
approche de \emph{validation de hold-out} pour évaluer la performance du
modèle. Notre échantillon d'apprentissage contient 70 \% des données
tandis que l'échantillon test contient les 30 \% restants.

\begin{verbatim}
[1] 67  9
\end{verbatim}

\begin{verbatim}
[1] 29  9
\end{verbatim}

\hypertarget{regression-sur-composantes-principales}{%
\subsection{Regression sur Composantes
principales}\label{regression-sur-composantes-principales}}

\begin{verbatim}
Data:   X dimension: 96 8 
    Y dimension: 96 1
Fit method: svdpc
Number of components considered: 8

VALIDATION: RMSEP
Cross-validated using 10 random segments.
       (Intercept)   1 comps   2 comps   3 comps   4 comps   5 comps   6 comps
CV        0.001518  0.001339  0.001239  0.001241  0.001239  0.001274  0.001141
adjCV     0.001518  0.001342  0.001237  0.001238  0.001234  0.001270  0.001132
        7 comps   8 comps
CV     0.001185  0.001180
adjCV  0.001175  0.001169

TRAINING: % variance explained
       1 comps  2 comps  3 comps  4 comps  5 comps  6 comps  7 comps  8 comps
X        35.21    64.81    78.84    89.59    94.05    97.80    99.87   100.00
Faits    24.33    35.89    37.54    42.38    42.80    55.19    56.96    57.59
\end{verbatim}

Après analyse des sorties du modèle PLS, on peut conclure que le modèle
explique environ \emph{57,6\%} de la variance de la variable dépendante
\emph{Faits} en utilisant 8 composantes. Le modèle a été validé à l'aide
de la RMSEP avec une validation croisée à 10 segments aléatoires et le
modèle avec \textbf{7 composantes} présente la meilleure performance
avec une valeur de 0.001075 pour CV et 0.001071 pour adjCV.

\includegraphics{code_files/figure-latex/pcr validation plot-1.pdf}

\includegraphics{code_files/figure-latex/unnamed-chunk-2-1.pdf}

Les modèles avec plus de composantes peuvent potentiellement présenter
une sur-ajustement, c'est-à-dire qu'ils peuvent être trop complexes pour
les données disponibles, ce qui peut entraîner des prédictions moins
précises sur de nouveaux ensembles de données.

\includegraphics{code_files/figure-latex/coefs estimated-1.pdf}

Les coefficients pour les variables \emph{Mediane\_revenu\_dispo},
\emph{Homme\_sans\_diplome}, \emph{Femme\_sans\_diplome} et
\emph{Taux\_chomage}, sont tous négatifs, ce qui suggère que des valeurs
plus élevées de ces variables sont associées à des valeurs plus faibles
du nombre de faits de viloences signalées. De même, le coefficient
positif pour la variable \emph{Taux\_logements\_sociaux} et
\emph{Taux\_pauvrete} suggère une association positive avec la variable
dépendante.

Toutefois, ces coefficients ne réfletent pas l'effet direct de chaque
variable sur le nombre de faits de violences conjugales.

\hypertarget{moindres-carruxe9s-partiels}{%
\subsection{Moindres carrés
partiels}\label{moindres-carruxe9s-partiels}}

\begin{verbatim}
Data:   X dimension: 96 8 
    Y dimension: 96 1
Fit method: kernelpls
Number of components considered: 8

VALIDATION: RMSEP
Cross-validated using 10 random segments.
       (Intercept)   1 comps   2 comps   3 comps   4 comps   5 comps   6 comps
CV        0.001518  0.001203  0.001214  0.001147  0.001122  0.001112  0.001102
adjCV     0.001518  0.001200  0.001201  0.001139  0.001114  0.001105  0.001096
        7 comps   8 comps
CV     0.001116  0.001116
adjCV  0.001109  0.001108

TRAINING: % variance explained
       1 comps  2 comps  3 comps  4 comps  5 comps  6 comps  7 comps  8 comps
X        33.19    49.65    71.13    80.77    92.50    95.52    98.97   100.00
Faits    42.52    51.11    54.63    56.77    56.93    57.11    57.24    57.59
\end{verbatim}

On peut voir que l'erreur diminue au fur et à mesure que le nombre de
composantes augmente, mais le taux de décroissance ralentit à partir de
5 ou 6 composantes. La meilleure performance est atteinte avec 6
composantes, où l'erreur de validation croisée est de 0.001156.

\includegraphics{code_files/figure-latex/pls cross validation plot-1.pdf}

En ce qui concerne l'interprétation des coefficients, chaque coefficient
représente uniquement la relation entre la variable explicative et celle
de réponse. Par exemple, un coefficient négatif pour la variable
\emph{Homme\_sans\_diplome} indique une relation inverse entre cette
variable et la variable de réponse, c'est-à-dire que les régions avec un
taux plus élevé d'hommes sans diplôme ont tendance à avoir une valeur
plus faible pour le nombre de faits de violences conjuguales signalées.
De même, un coefficient positif pour la variable \emph{Taux\_pauvrete}
indique une relation directe entre cette variable et la variable de
réponse, c'est-à-dire que les régions avec un taux plus élevé de
pauvreté ont tendance à avoir une valeur plus élevée pour le nombre de
faits de violences conjuguales signalées.

\hypertarget{muxe9thodes-puxe9nalisuxe9es}{%
\subsection{Méthodes pénalisées}\label{muxe9thodes-puxe9nalisuxe9es}}

Les méthode de pénalisation sont des techniques pour régulariser notre
modèle linéaire et réduire le risque de surajustement (overfitting).

\hypertarget{ruxe9gression-ridge}{%
\subsection{Régression Ridge}\label{ruxe9gression-ridge}}

\includegraphics{code_files/figure-latex/ridge-1.pdf}

\begin{verbatim}
[1] 9.116526e-05
\end{verbatim}

On constate que la valeur du meilleur paramètre lambda qui minimise
l'erreur quadratique moyenne estimée par validation croisée est
\textbf{9.116526e-05}.

\begin{verbatim}
9 x 1 sparse Matrix of class "dgCMatrix"
                                  s0
(Intercept)             5.445660e-03
Mediane_revenu_dispo   -1.744907e-07
Homme_sans_diplome     -2.172236e-08
Femme_sans_diplome     -1.398194e-08
Taux_chomage           -3.256978e-04
Taux_pauvrete           1.931377e-04
Taux_logements_sociaux  2.481363e-05
Sexe_politique          4.677787e-04
Geographie              2.651185e-04
\end{verbatim}

On conclue ici qu'un coefficient négatif indique une \emph{relation
inverse} avec la variable dépendante, tandis qu'un coefficient positif
indique une \emph{relation directe} avec la variable dépendante. Plus la
valeur absolue d'un coefficient est élevée, plus grande est l'importance
de la variable correspondante pour la prédiction de la variable
dépendante. On constate qu'en raison de l'utilisation de la
régularisation dans le modèle de Ridge, certains de nos coefficients
sont très petits (\emph{Homme\_sans\_diplome} et
\emph{Femme\_sans\_diplome}) par rapport à d'autres. Cela est dû à la
pénalisation de la magnitude des coefficients qui est utilisée pour
éviter le \textbf{surajustement} et améliorer le modèle.

\begin{verbatim}
[1] 0.5660313
\end{verbatim}

Le R-carré s'avère être de 0.5660313. C'est-à-dire que le meilleur
modèle a été en mesure d'expliquer 56.6\% de la variation des valeurs de
réponse des données d'entraînement.

\hypertarget{regression-lasso-hitters}{%
\subsection{Regression Lasso-Hitters}\label{regression-lasso-hitters}}

\includegraphics{code_files/figure-latex/lasso-1.pdf}

\begin{verbatim}
[1] 4.537317e-06
\end{verbatim}

On constate que la valeur du paramètre lambda qui minimise l'erreur
quadratique moyenne estimée par validation croisée est
\textbf{4.537317e-06}.

\begin{verbatim}
9 x 1 sparse Matrix of class "dgCMatrix"
                                  s0
(Intercept)             4.809058e-03
Mediane_revenu_dispo   -1.524774e-07
Homme_sans_diplome     -4.044207e-08
Femme_sans_diplome      .           
Taux_chomage           -4.110808e-04
Taux_pauvrete           2.527320e-04
Taux_logements_sociaux  3.041160e-05
Sexe_politique          4.719454e-04
Geographie              2.741934e-04
\end{verbatim}

On remarque qu'avec l'utilisation d'une régression de Lasso Hitters,
certains de nos coefficients sont très petits
(\emph{Mediane\_revenu\_dispo} et \emph{Homme\_sans\_diplome}) par
rapport à d'autres. La variable \emph{Femme\_sans\_diplome} a quant à
elle été retirée du modèle. Encore une fois, cela est dû à la
pénalisation de la magnitude des coefficients qui est utilisée pour
éviter le \textbf{surajustement} et améliorer le modèle.

\begin{verbatim}
[1] 0.5750197
\end{verbatim}

Le R-carré vaut 0.5750197. C'est-à-dire que le meilleur modèle a été en
mesure d'expliquer 57.5\% de la variation des valeurs de réponse des
données d'entraînement.

Ainsi, la régression de Lasso Hitters apparaît être plus performante et
proposer un meilleur modèle que la régression Ridge.

\hypertarget{ruxe9gression-elastic-net}{%
\subsection{Régression Elastic Net}\label{ruxe9gression-elastic-net}}

La régression Elastic Net combine deux formes de pénalisation, Ridge et
Lasso.

\begin{verbatim}
  alpha lambda
1     0      0
\end{verbatim}

On remarque que les meilleurs alpha et lambda estimés sur les données
d'entrainement sont équivalents à 0.

\begin{verbatim}
9 x 1 sparse Matrix of class "dgCMatrix"
                                  s1
(Intercept)             6.727882e-03
Mediane_revenu_dispo   -2.406051e-07
Homme_sans_diplome     -2.264493e-08
Femme_sans_diplome     -1.150290e-08
Taux_chomage           -3.768758e-04
Taux_pauvrete           1.576194e-04
Taux_logements_sociaux  5.071512e-05
Sexe_politique          8.349731e-04
Geographie              5.242352e-04
\end{verbatim}

Avec l'application d'une régression Elastic Net, toutes les variables
sont conservées dans le modèle final. Cependant, les variables
\emph{Mediane\_revenu\_dispo}, \emph{Femme\_sans\_diplome} et
\emph{Homme\_sans\_diplome} ont des coefficients plus petits que les
autres variables.

\begin{verbatim}
[1] 0.5985996
\end{verbatim}

Le R-carré vaut 0.5985996. Donc le meilleur modèle a été en mesure
d'expliquer 59.8\% de la variation des valeurs de réponse des données
d'entraînement.

\hypertarget{choix-du-moduxe8le}{%
\subsection{Choix du modèle}\label{choix-du-moduxe8le}}

Afin de sélectionner le meilleur modèle nous étudions le tableau
récapitulatif suivant :

\begin{verbatim}
# A tibble: 5 x 3
  Test                     Rsquare RMSE   
  <chr>                    <chr>   <chr>  
1 Composantes principales  56.96   0.0010 
2 Moindres carrés Partiels 57.11   0.0011 
3 Ridge                    56.6    0.0097 
4 Lasso                    57.5    0.0096 
5 Elastic Net              59.86   0.00078
\end{verbatim}

Au vu des différents indiateurs, Elastic net apparaît être la meilleure
régression car son R-squared est le plus grand et sa RSME est minimisée.
Son R-squared est aussi supérieur à celui de notre modèle initial,
ainsi, le modèle final conservé est celui produit par la régression
Elastic Net.

\newpage

\hypertarget{partie-3-analyse-de-lendoguxe9nuxe9ituxe9}{%
\section{Partie 3 : Analyse de
l'endogénéité}\label{partie-3-analyse-de-lendoguxe9nuxe9ituxe9}}

L'endogénéité survient lorsqu'une variable explicative (endogène) est
corrélée avec les erreurs de régression. Cela peut entraîner des biais
dans les estimations des paramètres du modèle et rendre difficiles
l'interprétation et la validité des résultats obtenus. Il existe trois
sources d'endogénéité :

\begin{itemize}
\tightlist
\item
  \emph{les variables omises,}
\item
  \emph{les erreurs de mesure,}
\item
  \emph{la simultanéité}
\end{itemize}

Plusieurs approches peuvent être utilisées pour traiter l'endogénéité
dans un modèle économétrique dont celle des variables instrumentales et
des modèles à équations simultanées.

Les variables instrumentales sont des variables qui sont corrélées avec
la variable endogène, mais qui ne sont pas corrélées avec les erreurs de
régression. Elles sont utilisées pour estimer les paramètres du modèle
en remplaçant la variable endogène problématique par ses valeurs
prédites à partir des variables instrumentales. Cela permet de supprimer
les biais potentiels dus à l'endogénéité. Avec les modèles à équations
simultanées, les variables endogènes sont modélisées simultanément
plutôt que séparément, ce qui permet de prendre en compte les
interactions entre elles et de capturer les relations de causalité
simultanée. Dans notre cas, les équations sont difficiles à identifier,
la méthode des variables instrumentales serait donc plus adaptée afin de
résoudre l'endogénéité qui pourrait être due aux variables omises.

Enfin, nous nous sommes questionnés sur les variables omises qui
pourraient être ajoutées à notre modèle. Premièrement, les facteurs
culturels ne sont pas présents dans nos variables. Une variable sur la
religion aurait pu être pertinente et révéler des différences sur le
nombre de faits, en fonction des proportions de chaque religion au sein
des départements français. Deuxièmement, d'autres facteurs individuels
seraient appropriés à notre étude tels que :

\begin{itemize}
\tightlist
\item
  la santé mentale des individus : les victimes de violences sexuelles
  peuvent avoir besoin d'un soutien psychologique pour faire face aux
  traumatismes, mais l'accès à des services de santé mentale peut varier
  en fonction du lieu de résidence, du niveau de revenu ou de
  l'assurance maladie)
\item
  les antécédents de violences sexuelles : les personnes ayant été
  victimes de violences auparavant peuvent être plus susceptibles d'être
  victimes à nouveau. Cependant, tout comme le recensement des faits de
  violences sexuelles, les antécédents sont très difficiles à évaluer
  pour chaque département.
\end{itemize}

Nous avons réfléchi aux éventuels erreurs de mesure qui causées des
biais dans notre estimation comme l'erreur de mesure des violences
sexuelles. En effet, certaines personnes peuvent ne pas être conscientes
qu'elles ont été victimes de violences sexuelles ou peuvent hésiter à
signaler des cas de violences sexuelles.

\newpage

\hypertarget{conclusion}{%
\section{Conclusion}\label{conclusion}}

Pour conclure, dans cette étude sur les violences sexuelles, nous avons
d'abord analysé notre base de données de façon univariée et bivariée
afin d'orienter au mieux la suite de notre réflexion. Après-coup, nous
avons sélectionné notre modèle économétrique, pour cela, nous avons
testé différents modèles. Après avoir vérifier la spécification du
modèle, nous avons retenu le meilleur modèle qui est celui ayant validé
les test d'abscence d'autocorrélation, d'hétéroscédasticité. La présence
d'une hétéroscédasticité nous a conduit à mettre en ouvre un modèle à
correction d'erreur (MCE). Ce modèle présentait néanmoins des problèmes
de multicolinéarité. Pour cela, nous avons appliqué des méthodes de
réduction de dimension et de pénalisation. Parmi ceux ci, le modèle
optimal est celui proposé par \emph{elastic net}. Ce modèle est défini
par :

\[ Faits=6.72e-03 -2.4e-07*Mediane.revenu-2.26e-08*Homme.sans.diplome-1.15e-08*Femme.sans.diplome \]
\[ -3.76e-04*Taux.chomage+1.57e-04*Taux.pauvrete+5.07e-05*Taux.logements.sociaux+\]
\[8.34e-04*Sexe.politique+5.24e-04*Geographie+e \]

Enfin, nous avons discuté des potentiels sources d'edogénéité qui
pourraient exister dans notre modèle. La source prédominante
d'endogénéité serait l'omission des variables
\texttt{antécédents\ des\ violences\ familiales} et la
\texttt{santé\ mentale}. Notre étude pourrait être poursuite en
appliquant la méthode des variables instrumentales dans un modèle des
Doubles Moindres Carrés (DMC).

\newpage

\hypertarget{bibliographie}{%
\section{Bibliographie}\label{bibliographie}}

\hypertarget{donnuxe9es}{%
\subsection{Données}\label{donnuxe9es}}

\begin{enumerate}
\def\labelenumi{\arabic{enumi}.}
\item
  \href{https://www.data.gouv.fr/fr/datasets/bases-communale-et-departementale-des-principaux-indicateurs-des-crimes-et-delits-enregistres-par-la-police-et-la-gendarmerie-nationales/}{Nombre
  de faits de violences sexuelles physiques par département}
\item
  \href{https://www.observatoire-des-territoires.gouv.fr/mediane-du-revenu-disponible-par-uc}{Revenu
  médian disponible en euros}
\item
  \href{https://www.data.gouv.fr/fr/datasets/bases-communale-et-departementale-des-principaux-indicateurs-des-crimes-et-delits-enregistres-par-la-police-et-la-gendarmerie-nationales/}{Population}
\item
  \href{https://www.insee.fr/fr/statistiques/1893149}{Homme sans
  diplôme}
\item
  \href{https://www.insee.fr/fr/statistiques/1893149}{Femme sans
  diplôme}
\item
  \href{https://www.data.gouv.fr/fr/datasets/logements-et-logements-sociaux-dans-les-departements-1/}{Taux
  de pauvreté, taux de chômage, taux de logements sociaux}
\item
  \href{http://www.cartesfrance.fr/carte-france-departement/carte-france-departements.html}{Situation
  géographique}
\item
  \href{https://www.regions-departements-france.fr/pdf/departements/liste-des-presidents-de-departement-2020.pdf}{Sexe
  du président du conseil départemental}
\end{enumerate}

\newpage

\hypertarget{annexe}{%
\section{Annexe}\label{annexe}}

Contribution au projet

\textbf{Bill} : Tout le projet sauf les régressions de Lasso et Elastic
Net.

\textbf{Jeanne} : Tout le projet sauf les régressions des Moindres
Carrés Partiels, des Composantes Principales et de Ridge.

\end{document}
